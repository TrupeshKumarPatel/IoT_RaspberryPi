\section{Introduction \& Background}	\label{sec:introduction-background}
Voice always has multiple functionalities among the information flowing at the edge of the IoT network. It carries valuable content, reflects the conditions of the environment, and can be used to command other entities through acoustic actuators and phone calls.

Implementing voice systems at the edge of an IoT network typically faces challenges, namely that edge devices are always constrained by computing power, bandwidth contention, and energy consumption. Therefore, it is not feasible to implement a complete TCP/IP stack on each node and give all nodes the ability to connect to remote entities outside the local network.

IP-PBX (IP-based Private Branch Exchange) provides a comprehensive solution to address the aforementioned issues. As the prototype of IP PBX, the design of the traditional PBX system is to serve a private organization, in which both the geographic area and the communication connection are limited to a specific scope. Interconnections between internal phones are without cost, while only central office lines provide connections to the public switched telephone network (PSTN). This scheme meets our expectations for edge IoT communication -- internal communication does not occupy egress bandwidth, and some switcher servers still reserve the communication egress to the outside. Leveraging VoIP (Voice over IP) technology, IP-PBX has ported the PBX scheme to the Internet, replacing telephone lines with packet-switching networks. The IP-based paradigm offers better scalability and lowers the cost same as the Internet brings to other domains. 

Several mature implementations of the IP-PBX paradigm are available, including 3CX and Asterisk PBX. Asterisk is an open-source software package that can run all the PBX functions, usually on a Linux operating system platform. It contains the functions of PBX and some other additional features. Along with the essential telephony services, voicemail services, conference calling, interactive voice response, and call queuing are also provided by Asterisk. It also provides multi-party calling, display caller ID (display calling number). To interact with digital telephone equipment and analog telephone equipment, Asterisk needs the support of PCI hardware, the most famous of which is provided by the Digium platform. 

From the architecture perspective, Asterisk serves as a middleware function, connecting the underlying telephony technology and the upper-level telephony applications. Both PBX and IVR (Interactive Voice Response) functionalities are integrated within Asterisk. Using compatible PCI hardware, Asterisk supports traditional telephone lines, including TDM (Time Division Multiplexing), TI/El PRI/PRA\&RBS (Robbed Bit Signal) mode, analog telephone line/analog telephone (POTS), ISDN (Integrated Services Digital) Network) and BRI (Basic Rate) and PRI (Primary Rate). Also, due to the PCI hardware support feature, Raspberry Pi can be used to implement the Asterisk instance.