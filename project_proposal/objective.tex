{
	\bfseries\textit{Objective}---
	Asterisk PBX is a software implementation of Private Branch Exchange (PBX), a system used to handle telecommunication services, allowing VoIP (Voice over Internet Protocol) services on computers, including less powerful devices such as Raspberry Pi. However, using smaller single-board devices come with the limitation of computing power and storage capacity. As a result of these limitations are dropping packets, cutting off the callers while handling a more significant number of calls simultaneously. In this project, the authors propose running a distributed Raspberry Pi model over an ad-hoc network with an Asterisk PBX system to determine how well VoIP services work within a distributed single-board system. Then this network configuration will be tested for the maximum number of calls, packets delays, network jitters, end-to-end network latency, and bandwidth as work shared between multiple Raspberry Pi.
%	Asterisk PBX is a software implmentation of Private Branch Exchange, a system used to handle telecommunication services. This allows for the usage of Voice over IP services on computers, including less powerful devices running on Raspberry Pis.	A limitation of using weaker devices to run Asterisk PBX is that computers with little memory or weak CPUs can quickly be overwhelmed when having to deal with a large number of callers at once, cutting off callers and dropping many packets.	In this proposal, the work required to run Asterisk PBX will be distributed over an ad hoc network of Raspberry Pis, to determine how well VoIP services work within a distributed single board system.	This configuration will be tested for maximum callers that the network can handle, delays with high number of callers, and increaed latency as work is shared between the Raspberry Pis.
}