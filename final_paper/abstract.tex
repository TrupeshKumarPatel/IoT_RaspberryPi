\begin{abstract}
	\label{sec:abstract}
		Asterisk PBX~\cite{sangoma-technologies} is an implementation of IP-PBX software, a digital version of Private Branch Exchange (PBX) technology, meant for IP networks instead of traditional telephone networks. IP-PBX software allows for the setup of VoIP (Voice over Internet Protocol) communications on computers and mobile devices, and can be run on less powerful devices such as Raspberry Pi. However, using smaller single-board devices come with the limitation of computing power and storage capacity. As a result of these limitations, packets can be dropped at a great rate, cutting off VoIP callers when a Pi needs to handle a significant number of calls simultaneously. In this project, the authors provides test the Raspberry Pi's suitability for providing IP-PBX services by analyzing the packets delays, network jitters, end-to-end network latency, and bandwidth between an Asterisk PBX system installed on Raspberry Pi 4~\cite{cytron-technologies} and IP clients that are using VoIP as service. This analysis is performed by testing the system with clients at varying distances from the Raspberry Pi.
\end{abstract}