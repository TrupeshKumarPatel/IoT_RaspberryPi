\section{Future Work}
    While this may act as a basic foundation for proving the efficacy of the Raspberry Pi as a cheap means to implement IP-PBX software, there are several means of expanding upon this work.
    \subsection{Further Testing}
        More experiments can be performed to determine the limitations that Raspberry Pis can prove to have when running IP-PBX software. A major test would be stressing the Pi by seeing how many calls it can handle in situations where bandwidth is plentiful, with the CPU and RAM of the Pi being the major bottlenecks at that point. This same test could be performed using different models of the Pi to see how the difference in CPU and RAM can affect VoIP calls. Analyzing this data could provide a way to determine the best board to purchase based on how many calls are needed to be handled.
        
        A limitation of data for outside network calls restricts current analysis. Future studies could attempt to have separate Raspberry Pis running PBX software on different networks, while allowing VoIP calls to be made between them. Data could then be collected based on test calls between these IP-PBX instances. This would prove useful in analyzing the effectiveness of the Pi in making outside network calls.
    \subsection{Horizontal Scaling}
        A powerful characteristic of cheap computers such as a Raspberry Pi is their suitability for horizontal scaling. If an institution grows large enough that one Pi cannot handle the all of the institution's VoIP calls, there are two methods of scaling up the memory and processing power available for IP-PBX services. The first is to scale the system vertically; replace the Pi with a single, more powerful computer with more memory, and either sell the Pi or reuse it for some other purpose. The second is to scale the system horizontally; connect more Raspberry Pis with the original and have them share the workload of the VoIP calls. As Raspberry Pis are inexpensive, this allows for a cheaper alternative to buying a more powerful computer. Further works can delve into whether this is a possible technique to implement with IP-PBX software, and whether or not horizontal scaling provides many benefits for IP-PBX services.