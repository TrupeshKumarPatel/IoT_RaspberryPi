\section{Conclusion} \label{sec:conclusion}
    A single Raspberry Pi with 4GB of RAM is quite capable of running IP-PBX software without any noticeable issues. However, configuring the software to allow for VoIP calls can be quite complicated, and requires a great deal of outside instruction to fully understand. As well, a vital component of using the Raspberry Pi for these purposes is considering the network involved. If the Pi or VoIP devices are in areas with weak signal strength, jitter can grow rapidly, packet loss can begin to noticeably occur, connections can be dropped abruptly, and delays can become much more noticeable. If these effects occur, either due to weak signal strength or due to other possible issues, call quality can suffer dramatically. Consideration is required to install a usable and useful IP-PBX system for VoIP calls.