\section{Introduction \& Background}	
\label{sec:introduction-background}
Voices can have multiple functions among the information flowing at the edge of the IoT network. They can carry valuable content, reflect the conditions of an environment, and can be used to command other entities through acoustic actuators and phone calls. However, implementing voice systems at the edge of such networks typically faces challenges, namely that edge devices are always constrained by computing power, bandwidth contention, and energy consumption. Therefore, it is not feasible to implement a complete TCP/IP stack on each node and give all nodes the ability to connect to remote entities outside the local network.

IP-PBX (IP-based Private Branch Exchange) provides a comprehensive solution to address the aforementioned issues. As the analog equivalent of IP PBX, the design of the traditional PBX system is to serve a private organization, in which both the geographic area and the communication connection are limited to a specific scope. Connections between internal phones are without cost, while only central office lines provide connections to the public switched telephone network (PSTN). This scheme meets our expectations for edge IoT communication -- internal communication does not occupy egress bandwidth, and some switcher servers still reserve the communication egress to the outside. Leveraging VoIP technology, IP-PBX has ported the PBX scheme to the Internet, replacing telephone lines with packet-switching networks. The IP-based paradigm offers better scalability and lowers the cost same as the Internet brings to other domains. 

Several mature implementations of the IP-PBX paradigm are available, including 3CX~\cite{3cx_2022} and Asterisk PBX~\cite{sangoma-technologies}. Asterisk is an open-source software package that can run all PBX functions, along with some other functions, usually on a Linux operating system platform. Voicemail services, conference calling, interactive voice response, and call queuing are provided by Asterisk, along with the essential telephony services. It also provides multi-party calling and displaying of caller ID (display calling number). To interact with digital telephone equipment and analog telephone equipment, Asterisk needs the support of PCI hardware, the most famous of which is provided by the Digium platform. 

From the architecture perspective, Asterisk serves as a middleware function, connecting the underlying telephony technology and the upper-level telephony applications. Both PBX and IVR (Interactive Voice Response) functionalities are integrated within Asterisk. Using compatible PCI hardware, Asterisk supports traditional telephone lines, including TDM (Time Division Multiplexing), TI/El PRI/PRA\&RBS (Robbed Bit Signal) mode, analog telephone line/analog telephone (POTS), ISDN (Integrated Services Digital Network) and BRI (Basic Rate) and PRI (Primary Rate). Also, due to the PCI hardware support feature, Raspberry Pi can be used to implement the Asterisk instance. 

By using small single-board computers such as Raspberry Pi, any types of businesses can deploy the cost affordable IP based PBX system. Raspberry Pi is small and light weight device, and runs on low power consumption. Having it for single tasks like allowing for VoIP by running Asterisk PBX is a very easy and economical. However, because this device is smaller and lighter, it can only handle a limited number of tasks. This paper will analyze the boundaries of the Raspberry Pi by packets delays, network jitters, end-to-end network latency, and bandwidth of calls made at varying distances from the Pi to determine how effectively IP-PBX services can be run on such devices.
	
This paper is organized into specific sections. \hyperref[sec:setup]{Section \ref{sec:setup}} lists the required hardware and software involved with the experiments performed here.  \hyperref[sec:obstacles]{Section \ref{sec:obstacles}} explains the obstacles faced when setting up for the tests in this experiment. \hyperref[sec:analysis-methods]{Section \ref{sec:analysis-methods}} explains the protocols that Asterisk PBX is using to connect devices and forward calls, how the calls are impacted by different network issues, and what topology IP-PBX services utilize to connect devices. \hyperref[sec:experiments]{Section \ref{sec:experiments}} shows detailed description for each experiment performed. \hyperref[sec:future-work]{Section \ref{sec:future-work}} goes over different ways the material of this paper can be expanded upon in the future. Finally, \hyperref[sec:conclusion]{Section \ref{sec:conclusion}} concludes the paper, giving an overview of the results seen thus far.