\begin{abstract}
	\label{sec:abstract}
		Asterisk PBX is a software implementation of Private Branch Exchange (PBX), a system used to handle telecommunication services, allowing VoIP (Voice over Internet Protocol) services on computers, including less powerful devices such as Raspberry Pi. However, using smaller single-board devices come with the limitation of computing power and storage capacity. As a result of these limitations are dropping packets, cutting off the callers while handling a more significant number of calls simultaneously. In this project, the authors provides the proof-of-concept by analyining the packets delays, network jitters, end-to-end network latency, and bandwidth between an Asterisk PBX system (which is install on Raspberry Pi 4) and IP clients that are using VoIP as service. This analyis includes internet through wired, wireless, cellular wireless connection. 
%		In this project, the authors propose running a distributed Raspberry Pi model over an ad-hoc network with an Asterisk PBX system to determine how well VoIP services work within a distributed single-board system. Then this network configuration will be tested for the maximum number of calls, packets delays, network jitters, end-to-end network latency, and bandwidth as work shared between multiple Raspberry Pi.
\end{abstract}